\documentclass[11pt, a4]{article}
\usepackage{url}

\usepackage{endnotes}

\let\footnote=\endnote

\usepackage{helvet}
\usepackage{times}

\usepackage{sectsty}
\allsectionsfont{\sffamily} 

% \parindent0cm

% Don't number any sections or subsections
\setcounter{secnumdepth}{-10}

\begin{document}

\title{Open Source Art Again}
\author{{\Large Rob Myers}\\
\texttt{http://www.robmyers.org/}}
\date{}

\sffamily
\maketitle
\rmfamily


%\begin{abstract}
%  Abstract
%\end{abstract}

\begin{quote}
Art is a game between all people of all periods.
\end{quote}
--- Marcel Duchamp\cite{Bourriaud2002}.


\section{Introduction}

The concept of ``Open Source'' continues to inspire artists. But it is an intentionally vague concept that is often more confusing than enlightening. Unpacking this examines the practices that are associated with it, and the ethical questions that it obscures. This reveals strategies that are of relevence to contemporary artistic practice and can place artists at the heart of current issues of free speech and the law and technology of censorship. 


\section{Open Source}

The concept of Open Source comes from computer software development. Open Source software is software that everyone can recreate and modify. This requires public access to the software's source code. The source code for a piece of software is equivalent to the score or preparatory work for a piece of music, drama or art, and it is similarly required to recreate or modify the finished work. Closed Source software, as sold by corporations such as Microsoft, does not have its source code publicly available and trying to recreate or modify it is prohibited by law. 

Open Source was given its current definition in 1998\footnote{See: \url{http://opensource.org/history}} by hackers Bruce Perens \cite{Dibona1999} and Eric Raymond. For Raymond, the virtue of Open Source is its efficiency \cite{Raymond1999}. Open Source, he argues, can create better products faster than the contrasting Closed Source method of writing software. Many of the most successful software programs in use today, particularly on the Internet, have been produced in such a way\footnote{For example the Apache web server, the Firefox web browser, the MySQL database and the GNU/Linux operating system. Of more interest to visual artists may be the Gimp image editor, Inkscape illustration program and Scribus DTP package.}.

The success of Open Source for software development in an age where computers are the defining technology and guiding metaphor of society\cite{Bolter1984} has inspired individuals and groups to try to apply its ideas to other areas of activity such as encyclopaedias, cartography, political activism, philosophy, theory, and art. Yochai Benkler describes this general application of Open Source as ``commons based peer production'' \cite{Benkler2006}; work made collaboratively and shared publicly by a community of equals.  This has often proved more problematic than might be expected. The idea of Open Source as a more efficient means of production doesn't explain why we should want to make philosophy or art more efficient, or what the form or advantage of that efficiency would be.

To take the example of the Open Congress event held at Tate Modern in 2005\footnote{See: \url{http://www.tate.org.uk/onlineevents/archive/open_congress/}}, artists struggled to find an Open Source ideology to apply to their art, activists struggled to find an Open Source ideology to apply to their organisations, and critical theorists invoked Deleuze and Spinoza to try to fill the gaps. There was a genuine interest in the potential of Open Source, but frequent confusion over what that might actually mean outside of software development.

The problem is that the name "Open Source" was deliberately chosen for its meaninglessness and ideological vacuity \footnote{Although there is some evidence that a previous use of the phrase ``Open Source'' dates back to at least the late 1980s.}. This was intended to make the results of a successful new ideology more palatable to large corporations by disguising its ethical content. That ideology is Free Software.


\section{Free Software}

American computer programmer Richard Stallman articulated the modern concept of Free Software in 1984 \cite{Stallman2002} in an environment of increasing restrictions on the use and production of software. Free Software is not a radical new development, it is a programme of reform intended restore the more freewheeling software development ethic that began to disappear when software became copyrightable in the USA in the late 1970s. 

Software is used to achieve many different ends within pluralistic society. Its use is as widespread and diverse as the written word was following the invention of the printing press. Free Software can therefore be understood historically and ethically as the defence of pluralistic freedom against a genuine threat. It is an ethical issue, a matter of freedom. This is very different from being a new method of organization or a more efficient means of production.

Stallman defines Free Software as a set of Four Freedoms \cite{Stallman2002}; the freedom to use, study, modify and share software. These freedoms are indivisible. If you have all four freedoms then you are free to use software. If you have only two or three then you are not. There are other definitions of Free Software\footnote{Such as the Debian Free Software Guidelines, see: \url{http://www.debian.org/social_contract}} but they tend to add redundant or contradictory terms and are best avoided. 

Once software users' freedoms are restored and protected, in the future by legal reform but until then by measures such as the GNU General Public Licence \footnote{See: \url{http://www.gnu.org/licenses/quick-guide-gplv3.html}}, this has the effect of encouraging collaborative and public software development. Licences like the GPL are a good example of measures that have been imitated outside of the context of software development. Collaborative, public software development protected by such licences is more efficient and can achieve higher quality than Closed Source ``proprietary'' software development. It is these effects of the pursuit of freedom that Open Source focuses on at the expense of obscuring the very causes that produce them.

It is important to remember that these gains, and the existence of commons-based peer production, are effects or products of freedom and the protection of people's rights. Without the guiding principles of the ethics of Free Software the neccessity and direction of Open Source projects cannot be accounted for. Open Source cannot account for itself or suggest which tasks are neccessary or important. It may become tempting to compromise some of the Four Freedoms to increase the efficiency or quality that are Open Source's promises, or in the name of political or economic goals.

This kind of confusion has led to projects such as Wikipedia trying to create an open space for anyone to use as they wish. This leads to Social Darwinism, not freedom, as the contents of that space is determined by a battle of wills. Wikipedia has had to reproduce many of the organizational structures and mechanisms of established Free Software projects in order to tackle these problems. But people still regard its earlier phase as a model for emulation, whereas it should probably serve as more of a warning.

Within Free Software projects, if contributions are deemed to be of acceptable quality, they are added to the project's source code by its appointed gatekeepers. If not, they are rejected and advice given. This methodology is a structured and exclusive one, but it is meritocratic. Any contribution of sufficient quality can be accepted, and if someone makes enough such contributions they themselves may gain the trust required to become a gatekeeper. This hierarchy and the decisions process is public and usually transparent, and if any individuals do not agree with the governance of the project they are free to take a copy of the source code of the project and start their own ``fork'' of it.

Free Software projects have produced commons based peer production only because of their practical pursuit of freedom. As with the term Open Source, the concept of commons based peer production easily confuses and misleads. Failing to account for the causes and effects of the pursuit of freedom can render commons based peer production inexplicable and in need of economic or political induction. This can lead to limits on the very freedom that actually produces the commons in order to protect investment (for neoliberal or robust individualist complainants) or class interests (for palao-socialist complainants). It is important to not compromise the pluralism produced by the pursuit of freedom in order to answer non-pluralistic concerns. To do so becomes self-defeating because compromising the freedom of others denies value from them to oneself.

There are strategies from Free Software that cannot work for art. Software can be replaced with a different, functionally equivalent, piece of software that does the same thing without loss. It is fungible. Art is not fungible. A Picasso is not a substitute for a Matisse artistically. This is not fetsishism, it is a product of the fact that an artwork's construction is what it does, you cannot change an artwork without changing its effect. The Free Software strategy of creating replacements for un-free work therefore cannot work for art. This means that there can be no repleacement for Modernist art that will remain covered by copyright and that these limitations must be dealt with through other means.

It is clear that ``Open Source'' obscures the very concept of freedom that leads to the beneficial effects identified by Raymond. ``Commons-based peer production'' is no more useful. It is a market-rational economic description of what for many economists seems to be the irrationality of the epiphenomena of Free Software. Again like Open Source it discards the ethical concepts that lead to its own production.

For all these reasons it is therefore the condition of Freedom rather than the condition of Open Source that art should aspire to.


\section{Artistic Freedom}

Freedom is the principle from which the organizatinal and economic benefits of Open Source flow. In applying the ideas of Free Software, to art we should look first not to the effects of pursuing freedom for software users but to what freedom means for art. To map the concepts of Free Software into artistic practice the best starting point is to identify the concept or concepts of artistic freedom and to identify the threats against it and opportunities for it.

Freedom applies to individual human beings, not objects. Artists have always learnt from, imitated, and built on from the work of their peers and of previous generations of artists. The history of art consists in no small part of the study of genres, schools, iconography, and studios involving many different individuals over many generations. Society has always provided both an audience and inspiration for art in a reciprocal relationship. Successive generations of artists, and artists and society, have collaborated to make the canon of art.

Prior to the extension of copyright to cover art as well as literature, art was implicitly free. The physical artefacts of art were expensive to own and difficult or impossible to transport. But the content of art was free to use by other artists and for critics and commentators to critique. This representational freedom of artists, part of which is the freedom to depict and build or comment on existing culture, to continue the conversation of culture, is in no small part artistic freedom.

Generations of artists could riff on the theme of the cruxifiction, and anyone could carve a statue of Venus. The production of Homeric verse was a multi-generational collaboration, and Shakespeare was a notorious plagiarist (or appropriator). Michaelangelo could appropriate christian and pagan imagery to paint a ceiling:
\begin{quote} 
the chapel’s funny shape — it has the basic outline of a treasure chest in a pirate movie — which was copied from an obscure Christian cartographer called Cosmas, whose chief claim to distinction is that he refused to accept that the earth was round. Cosmas insisted the earth was rectangular, and shaped, as it turned out, exactly like the Sistine Chapel.
\end{quote} \cite{Januszczak2005}. 

Historically there are examples of artworks, such as Raphael's ``Judgement Of Paris'', 1515, that we only know from unauthorized copies. Had reproduction of those works been prevented, they would be lost. The same student of Raphael that made these copies won a lawsuit that Albrecht Durer brought against him in 1506 for copying Durer's etchings\cite{McClean2002}. This predates the start of the modern concept of copyright by a hundred years.

Marcel Duchamp addressed material and organizational limitations on the creation of art in one of his interviews:
\begin{quote}When Rubens, or someone else, needed blue, he had to ask his guild for so many grams, and they discussed the question, to find out if he could have fifty or sixty grams, or more.\end{quote} \cite{Cabanne1987}  It is important to address material and organisational limitations, to create opportunities and support for the creation of art.

Art has always suffered from censorship. Religious, political and cultural restrictions on what can be shown by art have been joined more recently by economic and technological limitations.  

Calls for religious, political and cultural censorship continue to be a threat to artistic freedom. Opposing them should be a core part of the defence of artistic freedom. One of the weaknesses of simply imitating the strategies of Free Software is that software does not suffer from such calls, and so it has no strategies for coping with them. Given this, supporting legal reform and opposing more restrictive laws is vital. But supporting open critique is another way of tackling such calls, and Free Software does supply means of doing this.

Censorship through restrictive actions have been joined by restrictive law and now technological restrictions. The chilling effects of these on art are difficult to precisely quantify as they concern work not made and work not seen. But examples may illsutrate some of these effects.

Yves Klein patented the recipe for his International Klein Blue pigment in 1960\footnote{Allegedly French patent 63471.}. This would in theory have prevented anyone reproducing the precise colour for twenty years, limiting any artists who wished to comment on Klein's work or to take his ideas further.

Jeff Koons has lost several lawsuits from copyright holders for work that he has used as the basis of artworks, starting with the photographer of a postcard he based the sculpture  ``String Of Puppies'' on\footnote{See: Rogers v. Koons, 960 F.2d 301 (2d Cir. 1992)}, although he has more recently succeeded in a lawsuit from the copyright holders of photographs that he used as source material for the painting ``Niagara'' \footnote{Blanch v. Koons, No. 03 Civ. 8026 (LLS), S.D.N.Y., Nov. 1 2005 }. 

Andy Warhol lost a lawsuit from Patricia Caulfield, the copyright holder of the flower photograph that Warhol used as the source for his 1964  Flowers series\cite{McClean2002}. The Warhol Foundation allow use of Warhol's images for the creation of art and non-commercial use, although academics do still pay reproduction fees which seems to go against Warhol's own appropriationist spirit.

Warhol and Koons could afford such costs, although the time spent defending lawsuits is time that could better be spent making art. Less established artists can rarely afford the money or the time that legally defending their right to make art demands without assistance. 

Tom Forsythe's photographs of a Barbie doll in a blender led to a lawsuit from the corporate owners of the Barbie trademark. Tom fought the lawsuit with help from free speech charity The American Civil Liberties Union. His eventual victory inspired a project to celebrate it\footnote{See: \url{http://www.barbieinablender.org/}}. 

Painter Joy Garnett used an image that she found on the internet as inspiration for a painting ``Molotov'', 2004. After exhibiting it she found herself the subject both of legal threats from lawyers acting for the photographer's agency and of a campaign of support on the Internet involving making derivative images of Garnett's own derivative of the image\footnote{See: \url{http://www.firstpulseprojects.net/joywar.html}}. 

Artists can cause restrictions as well as suffering from them. Photography of Anish Kapoor's ``Cloud Gate'', 2005, was forbidden by the owners of the plaza in Chicago that it was installed in \footnote{See: \url{http://newurbanist.blogspot.com/2005/01/copyrighting-of-public-space.html}}. Photography of Christo and Jean Claude's ``Gates'' in New York's Central Park was forbidden by the artists themselves\footnote{See: \url{http://blog.stayfreemagazine.org/2005/02/more_on_christo.html}}, although this led to an organized campaign of photography \footnote{See: \url{http://flickr.com/groups/79972013@N00/pool/}} and online parodies in protest.

Even sketching in museums can be affected. When security guards tell children to stop drawing, whether this is a correct application of museum policy or a side effect of a general environment of strong copyright, the effects of copyright have gone too far\footnote{See: \url{http://www.boingboing.net/2005/01/08/stop_sketching_littl.html}}.

As ``Joywar'' and the reaction to The Gates shows, when media or corporate interests, or even other artists, seek to limit artists and the public from creating their own visual representations civil disobedience is often the result. This is as it should be. Bad law cannot be the limit of society's forms. 

Artistic freedom is part of the more general category of freedom of speech within an Open Society\cite{Popper2002}. Freedom of speech often becomes ``freedom of expression'' in artistic discussion, although ``freedom of representation'' might be a more useful concept as art more often causes prblems because of what it depicts than how the artist was feeling when creating the depiction. 

Support for freedom of speech involves both opposing censorship and providing support for those who would otherwise struggle to have their voice heard. These are the negative and positive freedoms of speech to use Isaiah Berlin's terms\cite{Berlin2002}. They apply equally to art as to the written or spoken word.

One aspect of art's value to society is that art is free to find new ways of looking at the world. In order to do this, artists must be free to depict whatever it is neccessary to do so however it is neccessary to do so. They might be prevented from doing so by new laws, for example by laws against depicting trademarked or copyrighted images or objects. Or they might be prevented from doing so by technology, for example by anti-copying measures on electronic media. 

Calling such measures ``censorship'' is not historically accurate, and expanding the category of censorship beyond government silencing of opposition is contentious. But there is no word that better fits the prevention of art's creation of new forms by reactionary institutional elements through new legal and technological measures\cite{Atkins2006}. 

A major threat to artistic freedom, then, is censorship in its current legal and technological forms. This is a limit on how art can be created and received, a restriction of freedom. To restore this freedom will require legal reform in the long term, but in the short term artists can look to Free Software for strategies.

``Free Art'' would mean that artists and society are free to produce and deal with art. Free software is concerend with the freedom of computer users to use software. Use in this context means utilization, not exploitation, use value not exchange value. And it means freedom as a general principle, not something contingent or alienable. Freedom of use does not cover trying to use software to remove the freedom of others to use software any more than freedom of contract covers slavery. Use of software covers production of software as well, since software is used to make software.

Artistic freedom covers the ability of artists, audience, academics and critics to experience, comment on (verbally or in new art), study and produce art \footnote{Following Stallman, freedom always seems to come in fours.}. Limiting this freedom is censorship in its expanded sense. Opposing censorship in its expanded sense is the defence of artistic freedom. 

Many artists have internalised the Romantic myth of creative genius and see their work as apart from society and needing protection from it through strong copyright and other measures. They have forgotten their own process of learning, their own influences, the criticism and journalism that supports their reputations, and that their audience's attention is part of the value of their art. Artists are tenants of culture, to quote Nicolas Bourriaud quoting Michel de Certeau\cite{Bourriaud2002}.

Artists learn from other artists and depict the broader visual environment that is produced by society. Artists are supported by and will learn from critics, academics and theorists. To charge for them to provide this support or to otherwise restrict it is unfair. A backlash against academic image reproduction fees has recently resulted from this \cite{BAJ2007}. 

Restrictions on artistic freedom stifle art and reduce its value to society. Appropriation artists and artists who depict contemporary events and the contemporary visual environment will be the first to feel these restrictions. Such art is an ``interrogation of meaning''\cite{Miller2004} and will generate precisely the kind of challenges to established or desired meanings that censorship is designed to preclude. These restrictions are being imposed from both within and without the artworld. Art must therefore be part of broader Free Culture both to defend itself and to avoid causing harm to others.

``Open Content'' is a simple mapping of the name Open Source onto cultural works. ``Content'' is what entertainment industry middlemen call production-line music and movies. Reviewers of disposable pop music use it as an insult \cite{Brooker2006}. A better term for creative work is ``culture'', and as we have established a more meaningful word than open is ``free''.

``Free Culture'' explains both the principle, freedom, and the subject of that principle, cultural as expressed through works and performances. Its advantages are the same as those of Free Software over Open Source.

The current understanding of Free Culture was popularized by lawyer and academic Lawrence Lessig's book of the same name\cite{Lessig2004}. Lessig founded the organization Creative Commons to address some of the issues he identified\footnote{See: \url{http://www.creativecommons.org/}}. Their imitation of Free Software's licencing tactic has led to a strong association between alternative licencing and the contemporary idea of Free Culture. 


\section{Licencing}

To protect the freedom of others to draw from and comment on your work is to protect your own freedom to do the same with their work. And with your own work, should you become alienated from it for whaever reason. Licences are a strong way of protecting this. Art as a whole has a social contract but the precise details vary between kinds of art. Appropriation art is the canary in the coalmine of artistic freedom. Once appropriation is restricted, criticism and then the freedom to depict the visual environment as a whole will be restricted.

Alternative licencing uses the tools of copyright licencing to add freedoms rather than impose restrictions. Once an alternative licence is applied to a copyrightable work, such as a piece of art, anyone who copies it or creates new work based on it is free to do so as long as they follow the licence. Ordinarily they would have no such freedom in law, outside of the bounds of Fair Use (or Fair Dealing).

Fair Use is mostly an American concept. Other legal systems have more limited exceptions to copyright, and would not allow artistic use of copyrighted imagery under the same terms. Alternative licences are useful for protecting Fair-Use style freedoms in such jurisdictions. Even within American law, Fair Use is a legal defence not a right, and must be defended in court if challenged. Against media corporations this will be a very uneven battle, and Lawrence Lessig has described Fair Use as little more in practice than``the right to hire a lawyer''\cite{Lessig2004}. So alternative licencing is useful for protecting Fair Use even within American law.

There are a number of different alternative licences available and several different kinds of licence. Some disallow commercial use. Since the work can be copied for free, and since the person licencing the work may want to use the results in turn, this isn't as useful as it might seem. Some disallow modifying the work, but the work can still be modified under Fair Use. Some simply allow the work to be copied and used with very few restrictions, but then the freedoms that make that possible can be removed in turn, shutting the audience and the original artist out.

The most successful licences, used by non-art projects such as GNU \footnote{See: \url{http://www.gnu.org/}} and Wikipedia \footnote{See: \url{http://wikipedia.org/}} allow work to be copied and new work to be based on it as long as people have the same freedoms over any copies or new work. This reversing of copyright to protect rather than remove people's freedoms was named ``copyleft'' by Stallman. 

The spread of work covered by copyleft licences produces a ``commons'' of work that people can draw from and contribute to freely. The historical metaphor of the commons comes from land owned \emph{and managed} by a community rather than a landowner. The often cited ``tragedy of the commons'', designed to prove that private ownership is better than common ownership, ignores this fact. 

It is important to remember that this cultural commons is a product of copyleft, which is a product of the ethical position of freedom. It is not an end in itself. Talk of the commons without talk of freedom can introduce broken metaphors from agricultural commons, or measures to protect the commons that would compromise the freedoms of individuals that give rise to the commons.

By relying on ever-strengthening copyright law, copyleft might appear to support the very thing it is designed to oppose. But this is not the case. If copyright law disappeared, copyleft would lose its force. Licencing is very popular with the ``Web 2.0'' internet bubble, where it is seen as a source of free labour for web sites and networks. Web 2.0 is an expression of the information knowledge work culture that Harold Liu identifies as opposed to the literary and artistic culture of history \cite{Liu2004}. Both the Web 2.0 and knowledge work objctions to licencing can be answered by the fact that is a form of intensification\cite{Harold2007} or ironisation of copyright law, a judo throw that uses its opponent's own weight against it. 

Licencing is not sufficient for artistic freedom, it is a measure against one specific threat to artistic freedom, the over-extension of copyright. Copyleft licences are also not an end in themselves and must change over time to best protect freedom against any new threats that emerge. New restrictions on freedom such as Digital Rights Management technology did not exist when the first copyleft licences were being drawn up, and modern copyleft licences do indeed tackle them. It is the content of the threat to freedom, expressed in any form be it legal or technological or other, that the form of copyleft licences must change to match in order to protect the content of freedom.

Copyleft was designed to protect the freedom to use computer software. But it can protect freedom of speech and artistic freedom as well. Due to the history of computer programming, programs are created and copyrighted as textual instructions to the computer. Copyleft is a means of removing the restrictions on freedom that copyright imposes not just on software but on any fixed form of expression. Since no small part of contemporary censorship is copyright-related, copyleft can be a useful means of addressing censorship where it can be applied.

Copyleft cannot protect freedom where it does not apply. If artists wish to work from art or media that are copyrighted but not licenced, they must fall back to Fair Use. If artists wish to depict objects or environments that are trademarks, there are not even licences for that at the moment. Broader reform and ongoing protection against more restrictive laws is therefore necessary, copyleft is not in itself a sufficient protection.

This also shows why art should share licences with other media rather than trying for specific licences. Doing so gives artists access in return to those parts of the media that adopt copyleft. And it also makes artists work available to critics, academics and other artists who wish to work with it. This can help drive awareness of the artists work, increasing their reputation and thereby the opportunities and remuneration available to them. 

Copyleft for software enforces the social contract of freewheeling hacker software development, but this may be different from the social contract of fine art. The art-specific Free Art Licence \footnote{See: \url{http://artlibre.org/licence/lalgb.html}} is based on copyleft. So is the more general (and more popular) Creative Commons BY-SA licence \footnote{See: \url{http://creativecommons.org/licenses/by-sa/3.0/}}. The only licence that attempts to enforce the social contract of a particular creative community is the Sampling licence from Creative Commons \footnote{See: \url{http://creativecommons.org/about/sampling}} and Negativland. The Sampling licence reflects the standards of the sampling and mash-up musical community, but it has some peculiarities, it is not considered ``Free'' by any common definition, and has since been deprecated by Creative Commons.

The social contract that the Sampling licence embodies is that of Extended Fair Use \cite{Negativland2005}. This allows transformative use and sampling but not wholesale copying. This is similar to the social contracts of appropriation art and of criticism. Copyleft goes further than this, allowing copying and incorporation of the work untransformed. For music, Negativland's focus, such wholesale copying is the hallmark of the peer-to-peer filesharing culture that has emerged following the release of Napster in 1999 and is now a freedom that is taken for granted by most music fans. So copyleft better fits the audience's expectations here.

To suggest that artists should apply copyleft to their work in order to take a position on artistic freedom in society might appear to replay twentieth century (and earlier) arguments about political commitment \cite{Arato1997}. ``It is always with the best intentions that the worst work is done.'', to quote Oscar Wilde, and the effects of political volunteerism by artists ranges from the negligible (Surrealism) to the negative (Socialist Realism). But for artists to protect artistic freedom is not volunteerism, it is key to maintaining the possibility of new forms in art. It is a practical response to the genuine threat of censorship in old and new forms. 


\section{Collaboration}

Collaboration and appropriation are ways of individuals building on the work of others. Collaboration can be local or distributed, parallel or serial. Collaborators working together at the same geographical location are local collaborators, those working over the internet or meeting only occasionally and otherwise working apart are disributed. Collaboration by a group of people on work at the same time is parallel collaboration, collaboration on a series of revisions of that work over time is serial collaboration, which is also a waye of describing appropriation.

Successful projects have strong social contracts. The existing social contract for art is more similar to Negativland's concept of extended fair use\cite{Negativland2005} than to copyleft. Copyleft is not a match for this but it is a superset of this, and (fine) art objects cannot be replaced by electronic copies of the original. 

The Surrealist drawing game ``exquisite corpse'' is local serial collaboration, collaborative projects like GNU and Wikipedia are distributed parallel collaboration. Appropriation and critique are distributed serial collaboration. 

Participation is what people generally mean when they say that a project or community is ``open''. People from outside the core of the project can join in and contribute to it. People can join the community without onerous membership tests. Participation does not neccessarily mean collaboration, and participation does not in itself guarantee freedom. 

It is possible for a group of collaborators to be closed to new members or external collaboration. Any pair or group of artists who collaborate privately do not have a participatory practice.

It is possible for a project to be participatory without being collaborative. Social networking sites such as Facebook\footnote{See: \url{http://www.facebook.com/}} allow people to participate without necessarily collaborating on projects.

Successful projects have strong social contracts. The existing social contract for art is more similar to Negativland's concept of extended fair use\cite{Negativland2005} than to copyleft. Copyleft is not a match for this but it is a superset of this, and (fine) art objects cannot be replaced by electronic copies of the original. 

Collaboration thrives when collaborators know that they will remain free to use the products of their collaboration. Wikipedia and GNU protect this social contract through their licences. The best way of signalling to potential collaborators that their freedom will be protected is to use such a licence. It isn't neccessary to be planning to collaborate in order for such licences to be useful. Finished work released into the world with such a licence makes the work of appropriation and critique easier.

It is possible to organize collabration without respecting the freedom of collaborators. Corporate ``user generated content'' initiatives that take copyright from contributors in return for the possibility of a prize remove the freedom of contributors to use their own work. Lawrence Lessig calls this ``sharecropping''\footnote{See: \url{http://www.washingtonpost.com/wp-dyn/content/article/2007/07/11/AR2007071101996.html}}. Projects that deny participants commercial use of their own work have the same effect. And people who come to projects complaining that if only the licence was changed they'd be able to use it in a way that excludes its creators are wannabe ``free riders'' by their own, economic, point of view and so should be ignored.

Wikipedia \footnote{See: \url{http://wikipedia.org/}} is a project to collaboratively create an encyclopaedia usable by anyone via the Internet. The site is licenced under a copyleft licence, meaning that anyone can add to or use it as long as they don't prevent anyone else from using it. There was another project to create an online encyclopaedia that predated Wikipedia, the h2g2 project \footnote{See: \url{http://h2g2.com/}}, but that was not a freely usable project and as a result ultimately failed both commercially and in terms of popularity.

Voluntury collaboration is not anti-individualistic, despite the charge levelled at Wikipedia by Jaron Lanier that it amounts to a collectivist ``digital maoism'' \footnote{See: \url{http://www.edge.org/3rd_culture/lanier06/lanier06_index.html}}. Individuals can pursue their own ends within a supportive structure and thereby both add value to and receive value from that structure as a whole. But the value of both individual contributions and the structure as a whole can be lost if all contributions are accepted without evaluation or if the project succumbs to structurelessness\footnote{See: \url{http://flag.blackened.net/revolt/hist_texts/ structurelessness.html}}. The online collaborative literary project ``A Million Penguins'' is a classic example of the incoherence that can result from this\footnote{See: \url{http://www.amillionpenguins.com/}}.  

The community arts project Remix Reading\footnote{See: \url{http://www.remixreading.org/}} ran workshops and accepted contributions through its web site and assembled exhibitions in local arts venues curated from the results. Various student shows run by chapters of freeculture.org\footnote{See: \url{http://freeartshows.org/shows/past}} followed the Remix Reading model using the image sharing site flickr\footnote{See: \url{http://flickr.com/photos/sharingisdaring2006/}} to accept contributions and then curating shows both in the real world and in the virtual reality environment Second Life\footnote{See: \url{http://www.freeartshows.org/shows/past}}.

These are mostly collaborative exhibitions of free work (and some non-free work under other Creative Commons licences, see the web site Freedom Defined\footnote{See: \url{http://freedomdefined.org/}} for a good list of free licences), rather than works created collaboratively, although some appropriation and remix art was included. This is a good model for supporting and promoting the production of freely licenced artworks.

When considering collaborative production of work one of the most successful examples from the world of software development is the Linux kernel. Anyone can submit work to be included in the project, anyone can discuss work that is submitted, but submissions are only included in the central repository for the project afer evaluation by the project's leaders. Anyone can view that repository, but the project's leadership controls the release of official versions. This ensures both that value can be accepted and included from outside the project, and that substandard work and reworking of the project cannot get into the project or become associated with it. 

It is easy to see how this model maps onto collaborative art shows mentione dabove but finding evaluation criteria for collaborative production of individual artworks or series of artworks is a difficult challenge. The collaborative image making web site Kollabor8\footnote{See: \url{http://www.kollabor8.org/}} avoids this problem by allowing each image to be forked from any previous version of an image and the results to be chosen by the audience. 

There are more examples of artistic collaboration than might be imanagined. Any artistic duo is involved in a collaborative practice. Licencing protects both the freedom of other members of society to appropriate and critique work produced by such artists and the artists themselves to do so should their association end. 

Individual artists producing what they regard as finished works can support the ability of artists and critics to learn from, comment on, and build on their work with copyleft licencing. This helps anyone whose work might promote the artist. It also encourages and enables collaborations and uses of the work that could not be predicted to the artist and may be of unique value to art and to society. 


\section{Economies}

Eric Raymond describes the culture of Open Source as a ``Gift Economy''\cite{Raymond1999}. Again, this is a product of Freedom and cannot be substituted for or sustained without it. The concept of gift economies comes from Marcel Mauss's work in anthropology in the first half of the twentieth century\cite{Mauss2001}. 

In a gift economy, gifts are given with strong social expectations that gifts will be given in return. It is this social contract that people exposed to the idea of gift economies often forget, regarding gifts as (economically irrational) random acts of kindness where in fact they are more like the enforced sharing of alternative licences (law being a way of enforcing society's norms between strangers).

The free sharing of ideas and iconography in art is a kind of immaterial gift economy. Making a physical gift economy of art can be an interesting commentary on the materialism of the art worl even in the era of relational art, and is a good investment in reputation.

Free Documenta and various solo projects organized by Sal Randolph, \footnote{See: \url{http://salrandolph.com/text/8/some-experiments-in-art-as-gift}} are a good example of creating a physical gift economy. These include organizing a global day of events with artists giving away prints of artworks or producing hundreds of blue paintings. Although they may appear to be random acts of kindness they do make a demand of their audience, to reflect the economic and social relations of the artworld. Against the backdrop of the global art market that is no small demand.

Randolph and hisCHECK collaborators are giving away works and reproductions of works that have cost money to produce. The freedom to share reproductions of work electronically via the internet reduces the value of copies of easily digitized and reproduced work to almost zero. 

The music and movie industries have panicked in response to this, and have reacted with lawsuits and inrusive propaganda to people who advertise their work for free through sharing it. There is no evidence that they are reacting to a geniune threat or that their actions are affecting people's willingness to share.

The gift of near-zero-cost reproductions of recordings of work is not lost revenue, it is promotion. Artists no less than musicians do not make most of their work from selling reproductions of their work and where they do make money from reproductions. They make their money by playing live (residency and show fees, public art fees), private gigs (commissions ) and merchandise (prints and editions of physical artworks, deluxe or personalized recordings or photographs of performances). 

Perhaps in order to address such concerns, Creative Commons do provide a pseudo-copyleft licence that only allows non-commercial (NC) use, which might appear to protect artists against economic exploitation. But it allows peer-to-peer sharing of work, which content industry claims is the major source of their loss of revenue, and individuals can print or paint their own copies of work shared in this way. 

NC prevents the artist of the original work from using any downstream reworkings of their piece commercially without negotiations that may be unsuccessful. Copyleft in itself is a stronger disincentive to people who wish to simply exploit work without giving anything in return, and NC allows verbatim copying anyway. It discourages critique and thereby promotion. And it prevents artists from recovering the costs incurred in creating derivative works\cite{Negativland1998}. NC is seductive but ultimately self-defeating. 

To quote book publisher Tim O'Reilly, ``Obscurity is a far greater threat to authors and creative artists than piracy.''\cite{OReilly2002}. The Internet is an excellent tool for fighting obscurity and creating opportunities to build reputation, and reputation is key to making a living as an artist. Artists do not generally make a living from selling reproductions of their work, and if they do they can compete on the basis of the quality and authenticity of those reproductions (forged signatures tend to be fairly worthless).

Benkler also discusses the idea of a ``reputation economy'' \cite{Benkler2006}. But there is no neat split between a reputation economy and the cash economy that an artist's reputation impacts on. And it is a mistake to try to protect the cash value of an artist's work while they are building a reputation that will increase the value of that work until such time as it can be sold to a traditional middleman. 

The Internet has proved a boon for artists, providing means of networking, selling work, and discussing and learning about art globally. From community sites like Furtherfield\footnote{See: \url{http:://www.furtherfield.org/}} and Rhizome\footnote{See: \url{http:://www.rhizome.org/}} to commercial sites like Saatchi's YourGallery \footnote{See: \url{http://www.saatchi-gallery.co.uk/yourgallery/}} and listings sites like ArtInfo \footnote{See: \url{http://www.artinfo.com/}} the internet has provided new opportunities.

Unlike recorded music, electronic images of artworks are not a substitute for most artworks, rather they promote the original. An artist and blogger like Joy Garnett, whose paintings draw on media imagery under the American Fair Use doctrine, can place high-resolution images of her paintings online\footnote{See: \url{http://www.firstpulseprojects.com/joy-work.html}} without fearing that they will be taken by any potential collector as a substitute for buying the original.

Performance, installation and sculpture are unlikely to be affected by electronic copying of images either. Work that is created as electronic images, as video or as software might appear to be more at risk, but as Matthew Barney's Cremaster Cycle shows it is perfectly possible to sell limited editions of infinitely reproducible art. 

Attempts led by the music and motion picture industries to restrict how the internet can be used, as government and lobby group attempts to block ``undesirable'' content on the internet are censorship in its expanded sense (and in its traditional sense where governments are involved). They provide no benefit to artists and will harm both artists and society's ability to benefit from the new possibilities that the internet affords. 

Artists can play an important part in showing that there are substantial uses for the Internet that do not involve what its opponents call ``piracy''. And can gain economically from the internet whether financially or reputationally.


\section{Conclusion}

Artists are not being distracted by external demands of political commitment when they take on issues of cultural freedom. They are exemplars. Free art, a free culture, is of vital importance for a free society. This freedom may result in commons based peer production, gift economies, reputation economies, increased efficiency and higher quality. But it is important not to confuse the incidental effects of an ideology with its principles. It is these principles that artists should pursue.

It is important to avoid repeating the mistakes of Open Source when doing so:

\begin{itemize}
\item Start from ``Free Software Free Society'' and ``Free Culture'', not ``The Cathedral And The Bazaar''. 

\item Don't try to organise your organisation in an ``Open Source'' way. That methodology is for content, not structure.

\item Don't try to emulate early Wikipedia's world-writeability. Emulate the meritocratic Linux Kernel development model that Wikipedia is slowly coming to resemble instead.

\item Don't be afraid of matters of principle. Renaming ``Free Software'' to ``Open Source'' has cost the people who have done so the biggest software market in the US, as the military are much more comfortable with ``freedom'' than they are with ``openness''.
\end{itemize}

There are many ways in which artists can apply the lessons of Free Software to art:

\begin{itemize}

\item Artists should become familiar with the concept of artistic freedom, the contemporary status of censorship, and how to protect the former against the latter.

\item Artists should campaign to oppose the extension of copyright or trademark law and the reduction of fair use. Where there are opportunities to lobby for extended Fair Use (such as the Gowers Report in the UK in 2007) artists should make sure their voices are heard.

\item Artists should use copyleft licensing to ensure the free circulation of ideas. If the sale of reproductions of work is a concern, investigate services that sell reproductions and experiment with releasing fewer works under a copyleft licence rather than more works under restrictive licences.

\item Artists who are interested to do so can investigate the use of collaborative project management.

\item Artists who are interested to do so should produce work to show the value of fair use and the public domain.

\item Artists who are interested to do so should challenge copyright maximalists and censors by using mass media imagery and transgressive imagery.

\item Artists should use Free Software and free (or "open") file formats for accessibility, and help drive improvement of them.

\end{itemize}

Applying these concepts to art is neither Digital Maoism nor economic irrationality but an ethical and social stance against the censorious restrictions that threaten to harm art's continued freedom. 


\begingroup
\renewcommand{\enotesize}{\normalsize}
\theendnotes
\endgroup


\bibliography{freedom}
\bibliographystyle{plain}

\section {Licence}

Copyright \copyright{} 2008 Rob Myers.

This work is licenced under the Creative Commons Attribution-Share Alike 3.0 Unported License. To view a copy of this licence, visit \url{http://creativecommons.org/licenses/by-sa/3.0/} or send a letter to Creative Commons, 171 Second Street, Suite 300, San Francisco, California 94105, USA.

\end{document}
